a-probabilistic-model-for.pdf desinteressante
a-proposal-of-a.pdf vazio
allan-02-harmonizing.pdf é sobre harmonização automática, não análise
allan-05-harmonizing.pdf é sobre harmonização automática, não análise
bainbridge-invite.pdf é desinteressante




a-238-prather.pdf é útil
bod01:memory.pdf quase relevante


\section{Algoritmos}

Pattern-matching over chord templates is used in a-329-prather.pdf.


\section{Representação}

In a238-prather.pdf, pitch height is ignored, as are accidents, and
notes are represented as pitch classes only, ignoring extra
information presente in accidents. Meter, also, is widely simplified,
being reduced to a percentage of time a given pitch class is sounding
in a given measure.

\section{Segmentação}

The segmentation problem, in a-329-prather.pdf, is approached in a
simplified fasshion, with chords being assumed to change only in the
beats of a given measure, and tonal information spread between two
measures being ignored. bod01:memory.pdf proposes a memory-based
model, contrasting it with rule-based ones, with good results,
although his approach to segmentation in unrelated to harmonic
analysis, and thus not very relevant. 

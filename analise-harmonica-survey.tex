\documentclass{article}
\usepackage{icmc,amsmath}
\usepackage{graphicx}
\title{A Survey of Automated Harmony Recognition and Tonality
  Detection Techniques}
 
% Single address
% To use with only one author or several with the same address
% ---------------
%\oneauthor
%  {Author} {School \\ Department}

% Two addresses
% --------------
%\twoauthors
%  {First author} {School \\ Department}
%  {Second author} {Company \\ Address}

% Three addresses
% --------------
\threeauthors
  {First author with an exceptionally long name} {School \\ Department}
  {Second author} {Company \\ Address}
  {Third author} {Company \\ Address}

\begin{document}

\maketitle

%% artigo deve ter 4 ou 8 paginas

\begin{abstract}
The abstract should be placed at the top left column and should contain
about 150-200 words.

The problems of automated harmony recognition and tonality detection
have 
\end{abstract}

\section{Terminology}


% uma lista de poss�veis termos a ser definidos, depois reduzir pra s�
% os relevantes e defin�-los
meter
pitch
segmentation
tonality
harmony
hidden markov model
melody
neural network
expert system
pattern matching
chord
greedy algorithm
island growing
ornamental note
music/harmonic context
% todos

\section{algoritmos}
% passos
Pattern-matching over chord templates is used in
\cite{prather96:harmonic}, \cite{pardo02:algorithms}, \cite{pardo99:automated} and \cite{barthelemy01:figured}, with the
latter using a somewhat more complex model, possibly ignoring
ornamental notes. A rule-based expert system approach is used in
\cite{buzzanca01:rule}, but for music style
regognition. \cite{buzzanca02:supervised} uses a back-propagating
neural network with far better effect. \cite{wang97:generating} uses
three passes for chord identification, all of them assuming a constant
tonality, and consisting basically of pattern-matching on many
levels. \cite{temperley99:modeling} uses a sophisticated
preference-rule based expert system to identify the roots of the
chords in a piece, but it avoids the problems of tonality detection
and harmonic function identification.\cite{raphael03:harmonic} uses a
hidden Markov model for chordal analysis and tonal recognition. 


For tonality recognition, \cite{barthelemy01:figured} uses an ``island
growing'' approach, a greedy algorithm that is based on the fact that
every chord can be only part of a few tonalities and, detecting each
of those for every chord, tries to merge them when possible, detecting
a tonality change when a merge isn't possible.


\section{tipos de resultado}
% pedro

\section{Meter and Pitch Representations}
% pedro, passos
In \cite{prather96:harmonic}, pitch height is ignored, as are accidents, and
notes are represented as pitch classes only, ignoring extra
information presente in accidents. Meter, also, is widely simplified,
being reduced to a percentage of time a given pitch class is sounding
in a given measure. In \cite{buzzanca02:supervised}, meter is
represented as the logarithm of beat length, ignoring most complex
beats. \cite{barthelemy01:figured} claims to distinguish enharmonic
notes, even though the exact representation used is not
discussed. MIDI input is used in
\cite{wang97:generating} and
\cite{raphael03:harmonic}. \cite{pachet96representing} describes a
codification for polyphonic tonal music, but makes no concrete claims
towards its applicability in harmonic
analysis. \cite{pardo02:algorithms}, \cite{pardo99:automated} uses a
simplistic system, representing pitches only as pitch classes,
ignoring meter and all forms of context information, with the
intention of providing a basis for sutdy of better algorithms.

\section{segmentation}

% passos, pedro

The segmentation problem, in \cite{prather96:harmonic}, is approached
in a simplified fashion, with chords being assumed to change only in
the beats of a given measure, and tonal information spread between two 
measures being ignored. \cite{bod01:memory} proposes a memory-based
model, contrasting it with rule-based ones, with good results,
although his approach to segmentation in unrelated to harmonic
analysis, and thus not very relevant. \cite{barthelemy01:figured} uses
melodic and harmonic information to construct the segmentation
bottom-up, ie merging two adjacent segments if and only every sound of
the first belongs in the second or if the union (with or without
ornamental notes) of both segments can be exactly mapped onto a
chord. The process starts with minimal segments and ends only when no
more merges can be performed. \cite{pardo02:algorithms} and
\cite{pardo99:automated} perform segmentation backwards, by comparing
all possible (or likely, in another version of the algorithm)
segmentations, choosing the ones that better match chord templates.


\section{teoria dos conjuntos}
% marcos, givaldo

\section{teoria musical}
% givaldo

\section{testes e compara��o}
% passos

\section{Harmonic Context}
% marcos

%% escrever: o que � contexto musical

%% rever tradu��o para 'sa�da'
The recent research projects that develop algorithms for automated
harmonic analysis doesn't make use of the harmonic context. However
the problems in out are, in most of cases related to context.

The set of algorithms for automated chordal analysis developed by
\cite{pardo02:algorithms} and \cite{pardo99:automated} make minimal use of harmonic context. The
errors frequencies are classified and most of them is related to
harmonic context, specifically non-chord notes and cadence
recognition.  The authors agree that a deeper understanding about
voice leading and the knowledge about neighbor chords would help to
adjust chord-label weights.

The automated harmonic analysis method developed by
\cite{barthelemy01:figured} has a simple algorithm to process
non-chord notes that produces errors. It consider that every non-chord
note is followed by a conjunct movement. The authors show problems
related to ambiguities, non-regularity of harmonic rhythm,
non-universality of harmonic rules.

\bibliographystyle{plain}
\bibliography{analise-harmonica-survey}
%\nocite{*} 
\end{document}
 
http://www.aaue.dk/~niklas/ICMC/?page_id=3

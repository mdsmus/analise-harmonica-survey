\documentclass{article}
\usepackage{icmc,amsmath}
\usepackage{graphicx}
\usepackage[T1]{fontenc}

\title{A Survey of Automated Harmony Recognition and Tonality
  Detection Techniques}
 
% Single address
% To use with only one author or several with the same address
% ---------------
%\oneauthor
%  {Author} {School \\ Department}

% Two addresses
% --------------
%\twoauthors
%  {First author} {School \\ Department}
%  {Second author} {Company \\ Address}

% Three addresses
% --------------
\threeauthors
  {First author with an exceptionally long name} {School \\ Department}
  {Second author} {Company \\ Address}
  {Third author} {Company \\ Address}

\begin{document}

\maketitle

%% artigo deve ter 4 ou 8 paginas

\begin{abstract}
The abstract should be placed at the top left column and should contain
about 150-200 words.

The problems of automated harmony recognition and tonality detection
have 
\end{abstract}

\section{Introduction}
\label{sec:introduction}
% pedro

%%% kroger:  basicamente falar que mir is a hot topic
Music information retrieval .... Some research focus on music
retrieval by query-by-humming
\cite{hu02comparison,pickens02:polyphonic,cano02:use}, in the
development of tools for digital libraries
\cite{clausen00:proms,olson03:chopin,fujinaga02:digital}, in the
detection of musical patters
\cite{lartillot-nakamura03:discovering,dannenberg02:pattern,lartillot02:integrating}
that may help, among other things, to help identifying musical styles
\cite{dixon03:classification,pampalk03:exploring,rauber02:using}, and
in problems related to musical representation
\cite{lee02:representing,good00:representing,rol00:xml4mir}.

While the main goal in music information retrieval `` is not an
understanding of music. It is not to develop a better theory of music,
or even to analyze music. The goal is retrieval''
\cite{pickens01:survey}, others researchers seek a better
understanding of elements and musical processes trough the automatic
analysis by a computer. We agree with Mouton that ``only automatic
analysis of a large corpus of tonal music may eventually provide us
(the human) with insights on the very nature of tonal music''
\cite{mouton95:numeric}. Computer-based musical analysis is important
because it can bring new insights in this area, in the same way the
use of the computer in vision and problem solving brought new insights
these areas \cite{temperley99:modeling}.

%%% kroger: scoring = arranjo?
There are many practical aplications for automatic music analysis,
among then are automated scoring, detection of posible logical
mistakes in scores, database search, automatic accompaniment
generation, and statitical analysis of musical styles for automated
composition \cite{pardo02:algorithms,temperley99:modeling}. Raphael and
%%% kroger: passive voice sucks badly
Stoddard emphasize the kinds of searches that can be made, where

\begin{quote}
  the most obvious application of harmonic analysis in music
  information retrieval (MIR) would treat queries phrased in the
  language of harmony: What are the earliest examples of the use of
  German augmented sixth chords or Neapolitan chords? Which Beatles
  songs have deceptive cadences? Where can I buy the piece I heard on
  the radio with the harmonic progression I vi IV V I repeated many
  times? It is likely that such applications will be most useful to
  musicologists since the mere formulation of such queries requires a
  more sophisticated understanding of harmony than would be expected
  of an average music enthusiast. \cite{raphael03:harmonic}
\end{quote}

By definition, automated harmony analysis must be done completely by
computer programs, without human intervention. Typically the system
accepts as input musical data in some format (such as MIDI, scores in
symbolic or numeric format, and so on) and returns as result
%%% kroger: tipo de acordes em ingles?
information such as chords roots, chords types, or symbols indicating
the tonal function. In this paper the terms ``analysis'' and
``harmonic analysis'' will be loosely used to mean automated harmonic
analysis by computer.

\section{The problem}
\label{sec:problem}
%% pedro: dividir

A an�lise autom�tica � uma tarefa complexa por diversas raz�es; o
material musical � composto de uma grande variadade de informa��es
(timbre, notas, ritmo, din�mica, harmonia) e pelo fato de m�sica ser
um processo temporal, diferentemente de imagem, por exemplo
\cite{mouton95:numeric}. 

\section{Terminology}
% todos

% uma lista de poss�veis termos a ser definidos, depois reduzir pra s�
% os relevantes e defin�-los
meter
pitch
segmentation
tonality
harmony
hidden markov model
melody
neural network
expert system
pattern matching
chord
greedy algorithm
island growing
ornamental note
music/harmonic context

\section{algoritmos}
% passos
Pattern-matching over chord templates is used in
\cite{prather96:harmonic}, \cite{pardo02:algorithms}, \cite{pardo99:automated} and \cite{barthelemy01:figured}, with the
latter using a somewhat more complex model, possibly ignoring
ornamental notes. A rule-based expert system approach is used in
\cite{buzzanca01:rule}, but for music style
regognition. \cite{buzzanca02:supervised} uses a back-propagating
neural network with far better effect. \cite{wang97:generating} uses
three passes for chord identification, all of them assuming a constant
tonality, and consisting basically of pattern-matching on many
levels. \cite{temperley99:modeling} uses a sophisticated
preference-rule based expert system to identify the roots of the
chords in a piece, but it avoids the problems of tonality detection
and harmonic function identification.\cite{raphael03:harmonic} uses a
hidden Markov model for chordal analysis and tonal recognition. 


For tonality recognition, \cite{barthelemy01:figured} and
\cite{mouton95:numeric} use an ``island growing'' approach, a greedy
algorithm that is based on the fact that every chord can be only part
of a few tonalities and, detecting each of those for every chord,
tries to merge them when possible, detecting a tonality change when a
merge isn't possible. \cite{temperley04:bayesian} uses a bayesian
model for key recognition.


\section{Tipos de resultado}
% pedro

\section{Meter and Pitch Representations}
% pedro, passos

In \cite{prather96:harmonic}, pitch height is ignored, as are
accidents, and notes are represented as pitch classes only, ignoring
extra information presente in accidents. Meter, also, is widely
simplified, being reduced to a percentage of time a given pitch class
is sounding in a given measure. In \cite{buzzanca02:supervised}, meter
is represented as the logarithm of beat length, ignoring most complex
beats. \cite{barthelemy01:figured} claims to distinguish enharmonic
notes, even though the exact representation used is not discussed.
MIDI input is used in \cite{wang97:generating} and
\cite{raphael03:harmonic}. \cite{pachet96representing} describes a
codification for polyphonic tonal music, but makes no concrete claims
towards its applicability in harmonic analysis.
\cite{pardo02:algorithms}, \cite{pardo99:automated} uses a simplistic
system, representing pitches only as pitch classes, ignoring meter and
all forms of context information, with the intention of providing a
basis for sutdy of better algorithms.

alguns sistemas num�ricos foram elaborados para manter as
caracter�sticas da m�sica tonal como \cite{hewlett92:base40} e
\cite{oliveira01:codificacao}, sendo que esse �ltimo apresenta uma
solu��o bastante elegante baseada em um m�dulo 96, sendo f�cil de
mapear para a representa��o de m�dulo 12. N�s acreditamos que a
pesquisa em an�lise harm�nica poderia ser grandemente simplificada
pelo uso desses sistemas.

O problema de n�o levar em considera��o as diferen�as enarm�nicas �
que acordes que possuem a mesma sonoridade mas tem fun��o tonal
diferente n�o ser�o detectados, como o acorde de sexta alem�. Em
\cite{pardo02:algorithms}, por exemplo, um acorde de sexta alem� �
erroneamente classificado como um acorde de s�tima da dominante.  

\section{segmentation}
% passos

The segmentation problem, in \cite{prather96:harmonic}, is approached
in a simplified fashion, with chords being assumed to change only in
the beats of a given measure, and tonal information spread between two 
measures being ignored. \cite{bod01:memory} proposes a memory-based
model, contrasting it with rule-based ones, with good results,
although his approach to segmentation in unrelated to harmonic
analysis, and thus not very relevant. \cite{barthelemy01:figured} uses
melodic and harmonic information to construct the segmentation
bottom-up, ie merging two adjacent segments if and only every sound of
the first belongs in the second or if the union (with or without
ornamental notes) of both segments can be exactly mapped onto a
chord. The process starts with minimal segments and ends only when no
more merges can be performed. \cite{pardo02:algorithms} and
\cite{pardo99:automated} perform segmentation backwards, by comparing
all possible (or likely, in another version of the algorithm)
segmentations, choosing the ones that better match chord templates.


\section{teoria dos conjuntos}
% givaldo

\section{teoria musical (contexto)}
% givaldo

\section{testes e compara��o}
% pedro

\section{Harmonic Context}
% marcos

%% escrever: o que � contexto musical

The recent research projects that develop algorithms for automated
harmonic analysis doesn't make use of the harmonic context. However
the problems in output are, in most of cases related to context.

The set of algorithms for automated chordal analysis developed by
\cite{pardo02:algorithms} and \cite{pardo99:automated} make minimal
use of harmonic context. The errors frequencies are classified and
most of them is related to harmonic context, specifically non-chord
notes and cadence recognition.  The authors agree that a deeper
understanding about voice leading and the knowledge about neighbor
chords would help to adjust chord-label weights.

The automated harmonic analysis method developed by
\cite{barthelemy01:figured} has a simple algorithm to process
non-chord notes that produces errors. It consider that every non-chord
note is followed by a conjunct movement. The authors show problems
related to ambiguities, non-regularity of harmonic rhythm,
non-universality of harmonic rules.

\bibliographystyle{plain}
\bibliography{analise-harmonica,bib-geral,bib-outras}
%\nocite{*} 
\end{document}
 
http://www.aaue.dk/~niklas/ICMC/?page_id=3

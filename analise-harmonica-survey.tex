\documentclass{article}
\usepackage{icmc,amsmath}
\usepackage{graphicx}

\title{Paper Template for ICMC 2007}
 
% Single address
% To use with only one author or several with the same address
% ---------------
%\oneauthor
%  {Author} {School \\ Department}

% Two addresses
% --------------
%\twoauthors
%  {First author} {School \\ Department}
%  {Second author} {Company \\ Address}

% Three addresses
% --------------
\threeauthors
  {First author with an exceptionally long name} {School \\ Department}
  {Second author} {Company \\ Address}
  {Third author} {Company \\ Address}

\begin{document}

\maketitle

%% artigo deve ter 4 ou 8 paginas

\begin{abstract}
The abstract should be placed at the top left column and should contain
about 150-200 words.
\end{abstract}

\section{Terminologia}
% todos

\section{algoritmos}
% passos
Pattern-matching over chord templates is used in \cite{prather96:harmonic}. A
rule-based expert system approach is used in \cite{buzzanca01:rule},
but for music style regognition. \cite{buzzanca02:supervised} uses a
back-propagating neural network with far better effect.

\section{tipos de resultado}
% pedro

\section{representa��o de altura e ritmo}
% pedro, passos
In \cite{prather96:harmonic}, pitch height is ignored, as are accidents, and
notes are represented as pitch classes only, ignoring extra
information presente in accidents. Meter, also, is widely simplified,
being reduced to a percentage of time a given pitch class is sounding
in a given measure. In \cite{buzzanca02:supervised}, meter is
represented as the logarithm of beat length, ignoring most complex
beats. 

\section{segmenta��o}
% passos, pedro

The segmentation problem, in a-329-prather.pdf, is approached in a
simplified fasshion, with chords being assumed to change only in the
beats of a given measure, and tonal information spread between two
measures being ignored. bod01:memory.pdf proposes a memory-based
model, contrasting it with rule-based ones, with good results,
although his approach to segmentation in unrelated to harmonic
analysis, and thus not very relevant. 


\section{teoria dos conjuntos}
% marcos, givaldo

\section{teoria musical}
% givaldo

\section{testes e compara��o}
% passos

\section{contexto}
% marcos

%\bibliographystyle{plain}
%\bibliography{analise-harmonica-survey}
 
\end{document}
 
http://www.aaue.dk/~niklas/ICMC/?page_id=3

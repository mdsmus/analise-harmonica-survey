\documentclass{article}
\usepackage{icmc,amsmath}
\usepackage{url}

\hyphenation{nea-po-li-tan}

\title{A Survey of Automated Harmonic Analysis Techniques}
 
\oneauthor
  {Pedro Kr�ger, Alexandre Passos, Marcos Sampaio}
  {Genos---Computer Music Research Group\\ School of Music
   \\ Federal University of Bahia, Brazil \\
  \url{pedro.kroger@gmail.com}, \url{alexandre.tp@gmail.com}, \url{mds_ufba@yahoo.com.br}}

\begin{document}

\maketitle

\begin{abstract}
  The problem of using the computer for automated harmonic analysis
  has been approached from several different directions with varied
  degree of success. Techniques from artificial intelligence such as
  machine learning and pattern recognition have been used. The
  division of a given piece in harmonic significant segments is a
  crucial problem in the area of automated harmonic analysis, although
  it is still underrepresented in the literature. Also, most
  algorithms ignore relevant contextual clues and little has been done
  towards establishing standard benchmarks. This paper presents a
  comprehensive survey of the existing literature on the subject.
\end{abstract}

\section{Introduction}
\label{sec:introduction}

Music information retrieval has received increasing attention
recently. Some research focus on music retrieval by query-by-humming
\cite{hu02comparison,pickens02:polyphonic}, development of tools for
digital libraries
\cite{clausen00:proms,olson03:chopin,fujinaga02:digital}, detection of
musical patterns
\cite{lartillot-nakamura03:discovering,dannenberg02:pattern,lartillot02:integrating}
that may help, among other things, to identify musical styles
\cite{dixon03:classification,pampalk03:exploring,rauber02:using}, and
in problems related to musical representation
\cite{lee02:representing,good00:representing,rol00:xml4mir}.

While the main goal in music information retrieval `` is not an
understanding of music. It is not to develop a better theory of music,
or even to analyze music. The goal is retrieval''
\cite{pickens01:survey}, other researchers seek a better understanding
of elements and musical processes through automatic analysis by
computer. We agree with Mouton that ``only automatic analysis of a
large corpus of tonal music may eventually provide us (the human) with
insights on the very nature of tonal music'' \cite{mouton95:numeric}.
Computer-based musical analysis is important because it can bring new
insights in musical theory, in the same way as use of computer in vision
and problem solving has brought new insights in these areas
\cite{temperley99:modeling}.

There are many practical applications for automatic music analysis,
among them arranging, detection of possible logical mistakes in scores,
database search, automatic accompaniment generation, and statistical
analysis of musical styles for automated composition
\cite{pardo02:algorithms,temperley99:modeling}.

Functional harmonic analysis is a specific kind of musical analysis,
in which a musical passage is represented as a sequence of chords. By
definition, automated harmonic analysis must be done completely by
computer programs without human intervention. Typically the system
accepts musical data in a format such as MIDI or scores in symbolic or
numeric format, and returns information such as chord roots, chord
types, or symbols in roman numerals indicating tonal function.

This paper gives an overview of the current status of automated
harmonic analysis. To our knowledge, there is no comprehensive survey
on this subject. Scholz \cite{scholz05:funchal} offers a brief survey
of the state of the art until 2001. Mouton \cite{mouton95:numeric} and
Barth�lemy \cite{barthelemy01:figured} have a general but informative
survey on the topic. This paper focus on the main problems and
solutions of automated harmonic analysis with a discussion of their
effectiveness. Harmonic analysis from audio sources will be not
approached.

\section{The problem of automated harmonic analysis}
\label{sec:problem}

The problem of automated harmonic analysis might be understood by
dividing it into three sub-problems. The first problem is to divide a
given piece in segments with harmonic significance. The second is to
label each segment with a chord name. The last problem is to assign a
tonal function to each chord.

A few additional sub-problems might arise from the input format, such as
determining meter or a correct pitch spelling depending on its
suitability.

\section{Music representation in harmonic analysis}
\label{sec:representation}

A musical piece must be represented in a form suitable to automated
analysis before making any attempt to extract information. There is a
broad number of codifications in existence, focusing on performance,
sound synthesis, computer games, musical notation, braille notation,
folk songs, musical bibliography, and many others
\cite{selfridge-field97:beyond}.

Interestingly, the choice of a proper codification influences the
results obtainable from analysis. For example, chords that sound
the same but have different tonal function and spelling might not be
told apart using an algorithm that does not consider enharmonic
difference. For instance, in \cite{pardo02:algorithms} a German
augmented sixth chord is mistakenly identified as a dominant seventh
chord.

Most systems
\cite{prather96:harmonic,pardo02:algorithms,pardo99:automated,temperley99:modeling}
represent notes as pitch classes, ignoring enharmonic information. This
is somewhat emphasized by the fact that MIDI or MIDI-like input is
used
\cite{pardo99:automated,pardo02:algorithms,wang97:generating,raphael03:harmonic}.
Pardo \cite{pardo02:algorithms,pardo99:automated} also ignores meter
and all forms of context information with the intention to provide a
basis for study of better algorithms.

A few numeric representations for pitch
\cite{hewlett92:base40,oliveira01:codificacao} where developed to keep
the characteristics of tonal music. The one developed by Oliveira
\cite{oliveira01:codificacao} is particularly elegant. This
representation is not only capable of differentiating enharmonic notes
(for instance, D$\sharp$ is represented by 7 and D$\flat$ by 91) but
allows operations like calculation of tonal intervals, transposition,
inversion, and so on. It is modulo 96 based and can be easily
converted to the more common module 12 based representation.

The systems described in literature usually return information such as
keys \cite{chai.vercoe:detection,noland.sandler:key}, chords
\cite{temperley99:modeling,pardo02:algorithms}, figured bass
\cite{barthelemy01:figured}, or tonal function
\cite{pachet00:computer,Ulrich77IJ}. Most systems, as those from
\cite{pardo02:algorithms,pardo00:automated,temperley99:modeling}, use
notation like A$\flat$7 for chords. The system described in
\cite{pachet00:computer} outputs not only the chord names but some
functions used in jazz harmony like ``modal borrowing'' and ``tritone
substitution''. Ulrich \cite{Ulrich77IJ} also performs a functional
analysis of chords, indicating information such as ``tonic of G'' or
``substitute for x+2''.

Since the usual notation for chords is ambiguous some alternatives
were proposed. Harte et al. notation \cite{harte.sandler.ea:symbolic}
has tuples of chord intervals, like \texttt{c:(3,5)} for C major or
\texttt{d\#:(b3,5,b7,9)/5} for D$\sharp$m$^{7}_{9}$/A$\sharp$. Since this notation is too verbose Harte et
al. use shorthands such as \texttt{d\#:min9/5}. Kroger et. al
\cite{kroger06:processo} use a list with chord information in the
format:\\
\texttt{(<name> <type> <intervals> <inversion>)}\\
For instance, G7/B would be \texttt{(g major 7 1)}.

\section{Harmonic Context}
\label{sec:harmonic-context}

In harmony some structures like non-chord notes are better understood
in a particular context. For instance, a passing note is a non-chord
note that only can be identified if is approached and left by step in
the same direction, so it is only identifiable by considering
contextual information. Proper use of harmonic context helps to solve
problems such as ambiguities in chord analysis, identification of
non-chord notes, suspensions, and retardations.

The NUSO system \cite{mouton95:numeric} uses harmonic context to
define tonality through rules considering non-chord notes in
accompanied melody, Alberti bass, and baroque style. The MusES system
\cite{mouton95:numeric} uses relations between adjacent chords to
define the best possible tonality for a chord. Finally, the automated
harmonic analysis method developed by \cite{barthelemy01:figured} has
a simple algorithm to process non-chord notes, often producing errors.
It assumes that every non-chord note is followed by a conjunct
movement which is not always true.

Some projects don't take harmonic context into account. Harte et al.
\cite{harte.sandler.ea:symbolic} ignores key context in their model
for chord labels. Pardo \cite{pardo99:automated,pardo02:algorithms}
developed a set of algorithms for automated chordal analysis making
minimal use of harmonic context.

Barth�lemy et al. \cite{barthelemy01:figured} raise four issues
related to harmonic context. Ornamental notes and incomplete harmony,
ambiguities, non-regularity of harmony, and non-universality of
harmony rules. Pardo et al. \cite{pardo99:automated} classify error
frequencies, most of them related to harmonic context,
specifically non-chord notes and cadence recognition. They agree that
a deeper understanding of voice leading and knowledge about neighbor
chords would help adjusting chord-label weights.

\section{Proposed Solutions}
\label{sec:proposed-solutions}

Many techniques from artificial intelligence, such as machine learning
and pattern recognition have been used in automated harmonic analysis.
These may be subdivided in the expert-system approach, concerned with
computerized representation of harmonic rules; and the statistical
approach, that deduces said rules from probabilistic models.

\subsection{Expert System Approaches}
\label{sec:expert-syst-appr}

Since expert systems abound in the field of artificial intelligence it
is only natural that they should be the first attempt at codifying
musical theory in a format proper for automated analysis. 

Pattern matching over chord templates is used in
\cite{prather96:harmonic}, \cite{pardo02:algorithms}, and
\cite{pardo99:automated}. Barth�lemy \cite{barthelemy01:figured} uses
a somewhat more complex model, that can ignore ornamental notes. Wang
\cite{wang97:generating} uses pattern matching in three passes and
assumes that no modulation happens. Temperley
\cite{temperley99:modeling} uses a sophisticated preference-rule
system to identify the roots of the chords in a piece, but it avoids
the problems of tonality detection and harmonic function
identification.

Barth�lemy \cite{barthelemy01:figured} and Mouton
\cite{mouton95:numeric} use a greedy algorithm for tonality
recognition that is based on the fact that every chord can be only
part of a few keys.

\subsection{Statistical Approaches}
\label{sec:stat-appr}

While not specifically addressing the problem of musical context,
statistical models differ from their rule-based counterparts by
considering this information implicitly. This is accomplished by
inference of implicit information from already harmonized scores
during training. Too much sensitivity to the training data might be
problematic, since it may lead the algorithm to mistakenly analyze a
piece. For instance, after being trained on Bach chorales if the
system is exposed to an arpeggiated melody it might classify each note
of the arpeggio as a different chord
\cite{tsui02:_harmon_analy_using_neural_networ}.

Back-propagating neural networks are used by T'sui
\cite{tsui02:_harmon_analy_using_neural_networ} with good results,
analyzing bach chorales with over 90\% efficiency. A hidden markov
model is employed by Raphael \cite{raphael03:harmonic} and Noland and
Sandler \cite{noland.sandler:key}, with over 91\% accuracy for key
estimation. A Bayesian model is used by Temperley
\cite{temperley04:bayesian} for key estimation with 86.5\% accuracy.

\section{Segmentation}
\label{sec:segmentation}

The division of a given piece in harmonic significant segments is a
crucial problem in the area of automated harmonic analysis, although
underrepresented in the literature
\cite{pardo02:algorithms,byrd02:problems}.

The task of designing an effective algorithmic solution for the
problem of segmentation is difficult because the segmentation problem
isn't clearly formalized in the music theory literature.

This problem is of high computational complexity, since the number of
possible partition points in a given piece is proportional to the
number of notes. The number of possible segmentations is approximately
two to the power of the number of partition points. To reduce the
space of possible partitions one must consider the harmonic context of
each segment, which is difficult. Most systems make little use of
proper context, according to Barth�lemy \cite{barthelemy01:figured}.

Prather \cite{prather96:harmonic} approaches the problem in a
simplified fashion, segmenting only inside measures. Therefore a chord
is split if it spans more than a bar. Barth�lemy
\cite{barthelemy01:figured} segments bottom-up, starting from minimal
segments and merging them in a larger segment if possible. Pardo
\cite{pardo99:automated,pardo02:algorithms} performs segmentation
backwards, by comparing all possible segmentations and choosing the
ones that better match chord templates.

\section{Tests and benchmarks}
\label{sec:tests-benchmarks}

Pardo states that ``no researchers have published statistical
performance results on a system designed to analyze tonal music''
\cite{pardo02:algorithms} before his paper. This lack of data in the
literature makes it difficult to compare different systems. Also, there
aren't standard benchmarks to compare different algorithms and
results. In fact, only Pardo \cite{pardo00:automated} and Barth�lemy
\cite{barthelemy01:figured} published specific comparisons between
theirs and other's results. Pardo compares his results with
\cite{temperley99:modeling} while Barth�lemy compares his model
against \cite{maxwell92:expert}, \cite{pardo99:automated} and
\cite{temperley96:algorithm}. However, they are based on the results
published in papers and not on results from direct implementations,
which means that only the examples published by the authors can be
compared. 

\section{Conclusion and discussion}
\label{sec:concl-disc}

This paper presented a comprehensive overview of the current status of
automated harmonic analysis, focusing on the main problems with a
discussion of their effectiveness.

The problems of harmonic segmentation and harmonic analysis are not
fully solved. A more systematic approach to segmentation might yield
considerably better results. Musical theory lacks precise formalisms,
so developing effective algorithms based on it is non-trivial. The
lack of standard benchmarks contributes to this problem and makes the
comparison of existent methods difficult.

Although many aspects of music can be understood using mathematical
models such as set theory (or post-tonal theory), it is not clear in
literature if a mathematical approach would help in harmonic analysis.
Taneyev \cite{taneyev62:counterpoint} has used mathematics extensively
in his counterpoint treatise with a good degree of success. As far as
we know, there are not meaningful works in the literature of automatic
harmonic analysis where a mathematical model such as set theory is
used extensively.

\bibliographystyle{plain}
\bibliography{analise-harmonica,bib-geral,bib-outras,analise-harmonica-nao-tenho}
\end{document}
 
http://www.aaue.dk/~niklas/ICMC/?page_id=3

\nota{para passos: n�o consigo ver aonde}
Barth�lemy \cite{barthelemy01:figured} claims to distinguish
enharmonic notes, even though the exact representation used is not
discussed.

